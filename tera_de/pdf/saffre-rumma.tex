%% Generated by Sphinx.
\def\sphinxdocclass{report}
\documentclass[letterpaper,10pt,ngerman]{sphinxmanual}
\ifdefined\pdfpxdimen
   \let\sphinxpxdimen\pdfpxdimen\else\newdimen\sphinxpxdimen
\fi \sphinxpxdimen=.75bp\relax

\PassOptionsToPackage{warn}{textcomp}
\usepackage[utf8]{inputenc}
\ifdefined\DeclareUnicodeCharacter
% support both utf8 and utf8x syntaxes
\edef\sphinxdqmaybe{\ifdefined\DeclareUnicodeCharacterAsOptional\string"\fi}
  \DeclareUnicodeCharacter{\sphinxdqmaybe00A0}{\nobreakspace}
  \DeclareUnicodeCharacter{\sphinxdqmaybe2500}{\sphinxunichar{2500}}
  \DeclareUnicodeCharacter{\sphinxdqmaybe2502}{\sphinxunichar{2502}}
  \DeclareUnicodeCharacter{\sphinxdqmaybe2514}{\sphinxunichar{2514}}
  \DeclareUnicodeCharacter{\sphinxdqmaybe251C}{\sphinxunichar{251C}}
  \DeclareUnicodeCharacter{\sphinxdqmaybe2572}{\textbackslash}
\fi
\usepackage{cmap}
\usepackage[T1]{fontenc}
\usepackage{amsmath,amssymb,amstext}
\usepackage{babel}
\usepackage{times}
\usepackage[Sonny]{fncychap}
\ChNameVar{\Large\normalfont\sffamily}
\ChTitleVar{\Large\normalfont\sffamily}
\usepackage{sphinx}

\fvset{fontsize=\small}
\usepackage{geometry}

% Include hyperref last.
\usepackage{hyperref}
% Fix anchor placement for figures with captions.
\usepackage{hypcap}% it must be loaded after hyperref.
% Set up styles of URL: it should be placed after hyperref.
\urlstyle{same}

\addto\captionsngerman{\renewcommand{\figurename}{Abb.}}
\addto\captionsngerman{\renewcommand{\tablename}{Tab.}}
\addto\captionsngerman{\renewcommand{\literalblockname}{Quellcode}}

\addto\captionsngerman{\renewcommand{\literalblockcontinuedname}{Fortsetzung der vorherigen Seite}}
\addto\captionsngerman{\renewcommand{\literalblockcontinuesname}{Fortsetzung auf der nächsten Seite}}
\addto\captionsngerman{\renewcommand{\sphinxnonalphabeticalgroupname}{Non-alphabetical}}
\addto\captionsngerman{\renewcommand{\sphinxsymbolsname}{Sonderzeichen}}
\addto\captionsngerman{\renewcommand{\sphinxnumbersname}{Numbers}}

\addto\extrasngerman{\def\pageautorefname{Seite}}

\setcounter{tocdepth}{1}



\title{saffre-rumma}
\date{22.10.2018}
\release{18.8.0}
\author{saffre-rumma}
\newcommand{\sphinxlogo}{\vbox{}}
\renewcommand{\releasename}{Release}
\makeindex
\begin{document}

\ifdefined\shorthandoff
  \ifnum\catcode`\=\string=\active\shorthandoff{=}\fi
  \ifnum\catcode`\"=\active\shorthandoff{"}\fi
\fi

\maketitle
\sphinxtableofcontents
\phantomsection\label{\detokenize{index::doc}}


Lino Tera ist ein Programm zur Verwaltung eines Therapiezentrums.


\chapter{Einführung}
\label{\detokenize{intro:einfuhrung}}\label{\detokenize{intro::doc}}

\section{Patienten, Therapien und Akten}
\label{\detokenize{intro:patienten-therapien-und-akten}}
Schon in TIM wurden Therapien „Akten“ genannt. Die Akten eines
gleichen Patienten wurden in TIM über ihre Erstanfrage verknüpft.
Lino unterscheidet zwischen „Patienten“ und „Akten“.  Eine Akte wird
fakturiert an einen Partner, die „Fakturierungsadresse“, die auch eine
Organisation oder ein Haushalt sein kann.  Anders als der Zahler in
TIM ist in Lino Fakturierungsadresse immer ausgefüllt.

Jeder Patient wird nur einmal erfasst, aber ein Patient kann an
mehreren Therapien teilnehmen, im Laufe der Jahre oder zeitgleich.

Es gibt drei Arten von Therapien: Einzeltherapien (ET), Lebensgruppen
(LG) und Therapeutische Gruppen (TG).

Jede Therapie hat einen einzigen verantwortlichen Therapeuten. Wenn
dieser wechselt, kann man entweder eine neue Therapie starten oder die
bestehende verändern.

Eine Therapie kann mehrere Teilnehmer haben. Bei ET ist das eher die
Ausnahme (aber durchaus möglich), bei LG sind es eher wenige und
konstante Teilnehmer, bei TG können es viele sein und die
Teilnehmerliste kann sich ändern.

Lino unterscheidet zwischen Therapien und „Teilnahmen“. Eine Teilnahme
ist die Tatsache, dass ein bestimmter Patient an dieser bestimmten
Therapie teilnimmt. Diese Unterscheidung ist wichtig in für LG und TG.
Eine ET wird in Lino behandelt wie eine Therapie mit nur einer
Teilnahme.


\section{Meine Akten}
\label{\detokenize{intro:meine-akten}}
Die Tabelle „Meine Akten“ gibt es in zwei Versionen: für Therapeuten
und für das Sekretariat.

Der Verwalter einer Therapie kann ein anderer sein als der
Therapeut. Zum Beispiel für Therapien, deren Termine durch das
Sekretariat verwalten werden.

Der Patient wird im Sektretariat erfasst.

Die Akte (Therapie) wird durch den Therapeuten erstellt und verwaltet.


\section{Dienstleistungen sind Termine}
\label{\detokenize{intro:dienstleistungen-sind-termine}}
Das, was in TIM „Dienstleistung“ hieß, heißt in Lino eher „Termin“.
Ein Termin ist, wenn ein Therapeut (oder mehrere) sich mit einem (oder
mehreren) Patienten trifft.


\section{Therapien mit mehr als einem Therapeuten}
\label{\detokenize{intro:therapien-mit-mehr-als-einem-therapeuten}}
In Therapien mit mehr als einem Therapeuten müssen die Therapeuten
sich einigen, wer in Lino der Verantwortliche ist.  Die anderen
Therapeuten stehen als Cotherapeuten in der Liste der Teilnehmer.


\section{Dienstleistungsarten}
\label{\detokenize{intro:dienstleistungsarten}}

\section{Erstgespräche}
\label{\detokenize{intro:erstgesprache}}
In der offenen Sprechstunde kann sich der potentielle
Patient mit einem Therapeuten zu einem Erstgespräch treffen.  Dort
wird entschieden, bei welchem Therapeuten der Patient eine Therapie
startet.  Zu diesem Zeitpunkt existiert noch keine Akte.


\chapter{Therapeuten}
\label{\detokenize{the/index:therapeuten}}\label{\detokenize{the/index::doc}}

\section{Termine erfassen}
\label{\detokenize{the/index:termine-erfassen}}
Klicken sie in der Startseite auf \sphinxguilabel{{[}Termin erstellen{]}}, um
einen neuen Termin zu erfassen. Lino zeigt dann ein Dialogfenster, in
dem Sie angeben müssen:
\begin{itemize}
\item {} 
\sphinxstylestrong{Datum und Uhrzeit}. Wann der Termin stattgefunden hat (Beginn).

\item {} 
\sphinxstylestrong{Akte}. Alle Termine sollten einer Akte zugewiesen sein. Lino fügt
alle Patienten einer Akte als Gäste in den Termin ein.

\item {} 
Die \sphinxstylestrong{Dienstleistungsart} ist wichtig für Fakturierung und Statistik

\item {} 
Die \sphinxstylestrong{Kurzbeschreibung} ist ein freier Text.

\end{itemize}

Wenn Sie das Dialogfenster bestätigt haben, zeigt Lino den erstellten
Termin in Vollbild-Ansicht. Hier können Sie die obigen Angaben falls
nötig ändern, sowie weitere Angaben zu diesem Termin erfassen:
\begin{itemize}
\item {} 
\sphinxstylestrong{Beschreibung} : (zu klären, inwiefern es Sinn macht, hier Text
einzugeben)

\item {} 
\sphinxstylestrong{Anwesenheiten} : Lino fügt automatisch alle Patienten einer Akte als
Gäste in den Termin ein.

\item {} 
\sphinxstylestrong{Workflow} : Hier können Sie den Zustand des Termins sehen und
ändern. Ein Termin kann folgende Zustände haben: Vorschlag / Entwurf
/ Stattgefunden / Storniert / Verpasst

\end{itemize}


\section{Meine Termine}
\label{\detokenize{the/index:meine-termine}}
Wähle im Hauptmenü \sphinxmenuselection{Kalender \(\rightarrow\) Meine Termine}, um
alle Termine ab heute zu sehen und zu verwalten.  Hier kannst du auch
neue Termine einfügen (Button \sphinxincludegraphics{{insert}.png} in der Werkzeugleiste) oder
Fehleingaben korrigieren.


\section{Akten anlegen}
\label{\detokenize{the/index:akten-anlegen}}
Wähle im Hauptmenü \sphinxmenuselection{Akten \(\rightarrow\) Meine Akten}, um alle
Akten zu sehen, deren verantworlicher Therapeut Du bist.  Hier kannst
Du z.B. eine neue Akte einfügen (Button \sphinxincludegraphics{{insert}.png} in der
Werkzeugleiste) oder die Akte aussuchen, mit der Du arbeiten möchtest.


\chapter{Sekretariat}
\label{\detokenize{sek/index:sekretariat}}\label{\detokenize{sek/index::doc}}

\section{Buchhaltung}
\label{\detokenize{sek/ledger:buchhaltung}}\label{\detokenize{sek/ledger::doc}}

\subsection{Einkaufsrechnungen}
\label{\detokenize{sek/ledger:einkaufsrechnungen}}
Um eine neue EKR zu erfassen, wähle im Hauptmenü
\sphinxmenuselection{Buchhaltung \(\rightarrow\) Einkauf \(\rightarrow\) Einkaufsrechnungen} und
klicke dann auf \sphinxincludegraphics{{insert}.png} um eine neue Rechnung einzufügen.

Der Partner einer EKR ist der Lieferant. Das ist üblicherweise eine
Firma oder Organisation, kann aber potentiell auch eine Einzelperson
oder ein Haushalt sein.

Das Buchungsdatum ist fast immer das gleiche wie das
Rechnungsdatum. Ausnahme: Wenn eine Rechnung n einem anderen
Kalenderjahr gebucht wird, dann muss als Buchungsdatum der 01.01. oder
31.12. des Buchungsjahres genommen werden.

In Total inkl. MWSt. gib den Gesamtbetrag der Rechnung ein. Lino wird
diesen Betrag ggf im folgenden Bildschirm verteilen.

Tipp : tippe \sphinxcode{\sphinxupquote{Ctrl-S}}, um dieses Dialogfenster ohne Maus zu
bestätigen.

Hier hat Lino den Gesamtbetrag so gut es ging aufgeteilt. Im Idealfall
kannst du hier auf „Registriert“ klicken, um die Rechnung zu
registrieren. Und dann wieder auf um die nächste Rechnung einzugeben.

Alternativ kannst du Konto, Analysekonto, MWSt-Klasse und Beträge
manuell für diese eine Rechnung ändern.

Lino schaut beim Partner nach, welches Konto Einkauf dieser Partner
hat. Falls das Feld dort leer ist, nimmt Lino das Gemeinkonto
„Wareneinkäufe“. Das MWSt-Regime der Rechnung nimmt Lino ebenfalls vom
Partner. Beide Felder kannst du in den Partnerstammdaten nachschauen
gehen, indem du auf die Lupe (\sphinxincludegraphics{{search}.png}) hinterm Feld „Partner“
klickst. Dort kannst du diese beiden Felder dann für alle zukünftigen
Rechnungen festlegen.


\subsection{Analysekonten}
\label{\detokenize{sek/ledger:analysekonten}}
Analysekonten und Generalkonten sind zwei unterschiedliche
Klassierungen der Kosten. Der Buchhalter interessiert sich nur für die
G-Konten und weiß von den A-Konten nichts. Der VWR dagegen
interessiert sich eher für die A-Konten.

Über \sphinxmenuselection{Konfigurierung \(\rightarrow\) Buchhaltung \(\rightarrow\) Konten} kann
man den Kontenplan (d.h. die Liste aller Generalkonten) sehen und
ggf. verändern.

Pro Generalkonto kann man sagen :
\begin{itemize}
\item {} 
Braucht AK : wenn angekreuzt, dann muss für Buchungen auf dieses
Konto auch ein A-Konto angegeben werden. Wenn nicht angekreuzt, dann
darf für Buchungen auf dieses Konto kein A-Konto angegeben werden.

\item {} 
Analysekonto : welches A-Konto Lino vorschlagen soll, wenn man
dieses Generalkonto für eine Buchung auswählt.

\end{itemize}

NB das A-Konto des Generalkontos ist lediglich der Vorschlag
bzw. Standardwert. Man kann das A-Konto einer individuellen Buchung
manuell dennoch ändern.

Pro Generalkonto kannst du das Analysekonto angeben, das Lino
vorschlagen soll, wenn du eine neue Einkaufsrechnung (EKR)
eingibst. In der EKR kannst du dann immer noch ein anderes AK
auswählen. Du kannst das AK im Generalkonto auch leer lassen (selbst
wenn „Braucht AK“ angekreuzt ist). Das bedeutet dann, dass Lino in der
EKR keinen Vorschlag machen soll. Dann ist man sozusagen gezwungen,
bei jeder Buchung zu überlegen, welches AK man auswählt.
\subsubsection*{Tipps}

Pro Partner kannst Du das Konto Einkauf festlegen. Dieses Konto trägt
Lino dann automatisch als Generalkonto in Einkaufsrechnungen von
diesem Partner ein.

Nach Ändern des Generalkontos in einer Rechnungszeile setzt Lino immer
das Analysekonto, selbst wenn dieses Feld schon ausgefüllt war.


\subsection{Verkaufsrechnungen}
\label{\detokenize{sek/ledger:verkaufsrechnungen}}
Abgesehen von den automatisch erstellten Rechnungen kannst Du
jederzeit auch manuell Verkaufsrechnungen (VKR) erstellen und
ausdrucken. Manuelle VKR stehen üblicherweise in einem eigenen
Journal, um sie nicht mit den automatisch erstellten VKR zu
vermischen.

Um eine neue VKR zu erfassen, wähle im Hauptmenü :menuselection:{}`
Buchhaltung \textendash{}\textgreater{} Verkauf \textendash{}\textgreater{} (gewünschtes Journal){}` und klicke dann auf
\sphinxincludegraphics{{insert}.png} um eine neue Rechnung einzufügen.

Anders als bei Einkaufsrechnungen gibst Du in VKR keinen Gesamtbetrag
ein und wählst einen „Tarif“ statt eines Generalkontos. Ansonsten ist
die Bedienung ähnlich.


\subsection{Kontoauszüge}
\label{\detokenize{sek/ledger:kontoauszuge}}
Für jedes Bankkonto gibt es in Lino ein entsprechendes Journal. Für
jeden Kontoauszug der Bank gibst du einen Kontoauszug in Lino
ein. Achte dabei auf Übereinstimmung der Nummern sowie der alten und
neuen Salden. Jeder Kontoauszug wiederum enthält eine oder mehrere
Zeilen, je eine pro Transaktion.

Um einen neuen Kontoauszug zu erfassen, wähle zunächst im Hauptmenü
\sphinxmenuselection{Buchhaltung \(\rightarrow\) Finanzjournale} und dort das
gewünschte Journal. Lino zeigt dann eine Tabelle mit den bereits
erfassten Kontoauszügen. Hier kannst du sehen, wo du beim letzten Mal
aufgehört hast.

Doppelklick auf einem bestehenden Auszug zeigt dessen
Vollbild-Ansicht.

Klicke auf \sphinxincludegraphics{{insert}.png} in der Werkzeugleiste, um einen neuen Kontoauszug
einzufügen.

Als Buchungsdatum gib das Datum des Kontoauszugs ein. Der Alte Saldo
sollte der gleiche sein wie der Neue Saldo des vorigen Kontoauszugs.

Dieses Fenster kannst du auch per Tastatur mit \sphinxcode{\sphinxupquote{Ctrl+S}}
bestätigen.  Jetzt zeigt Lino den noch leeren Kontoauszug:

Hier musst du auf \sphinxincludegraphics{{own_window}.png} klicken, um das untere Panel in einem
eigenen Fenster zu öffnen.

Das Feld Nr füllt Lino automatisch aus. (Du kannst auf einer
bestehenden Nummer \sphinxcode{\sphinxupquote{F2}} drücken und sie ändern, um die
Reihenfolge der Zeilen zu beeinflussen).

Das Feld Datum kann leer bleiben, dann trägt Lino das Datum des
Kontoauszugs ein.

Wenn es sich um die Zahlung einer Rechnung handelt, muss im Feld
Partner der Kunde oder Lieferant ausgewählt werden. Lino schaut dann
nach, ob offene Rechnungen vorliegen und tut folgendes.

Wenn es genau eine offene Rechnung gibt, füllt Lino in den Feldern
Match und Einnahme bzw. Ausgabe die Zahlungsreferenz und den Betrag
der Rechnung ein.

Wenn es mehrere offene Rechnungen gibt, trägt Lino im Feld Match den
Text „x Vorschläge“ ein. Das bedeutet, dass du auf das Wort
\sphinxstyleemphasis{Vorschläge} klicken solltest.  Siehe weiter unten.

Wenn es keine offene Rechnung gibt, musst du die Felder Match und
Einnahme bzw. Ausgabe selber ausfüllen.

Wenn Lino den Betrag ausgefüllt hast, kannst du diesen trotzdem noch
abändern. Zum Beispiel bei Teilzahlung oder Zahlungsdifferenz.

Das Feld Partner bleibt leer, wenn es sich um eine allgemeine Buchung
(Generalbuchung oder partnerlose Buchung) handelt, die nicht an einen
bestimmten Geschäftspartner bezogen ist und nicht beglichen werden
müssen. Zum Beispiel interne Transfers von einem Bankkonto zum
anderen, Abbuchung von Zahlungsaufträgen, Bankunkosten,
Kreditrückzahlungen, Mieten, Zuschüsse.

Im Feld Konto kommt das Generalkonto zu stehen. Dieses Feld muss immer
ausgefüllt sein. Wenn du einen Partner ausgewählt hast, dann steht
hier eines der Konten „Kunden“ oder „Lieferanten“.


\subsection{Buchhaltungsberichte drucken}
\label{\detokenize{sek/ledger:buchhaltungsberichte-drucken}}
Menü \sphinxmenuselection{Berichte \(\rightarrow\) Buchhaltung \(\rightarrow\) Buchhaltungsbericht}.

Tipp: nachdem Du \sphinxstylestrong{Periode vom} (und optional \sphinxstylestrong{bis}) eingegeben
hast, musst du auf den Blitz klicken, damit Lino die Daten berechnet
und anzeigt.

Tipp: wenn Du \sphinxstyleemphasis{Periode bis} leer lässt, wird nur \sphinxstyleemphasis{Periode vom}
berücksichtigt.


\section{Patienten erfassen}
\label{\detokenize{sek/index:patienten-erfassen}}
Um einen neuen Patienten zu erfassen:
\begin{itemize}
\item {} 
\sphinxmenuselection{Kontakte \(\rightarrow\) Patienten}

\end{itemize}


\chapter{Anlageverwalter (Site-Administrator)}
\label{\detokenize{adm/index:anlageverwalter-site-administrator}}\label{\detokenize{adm/index::doc}}

\section{Benutzerverwaltung}
\label{\detokenize{adm/users:benutzerverwaltung}}\label{\detokenize{adm/users::doc}}

\subsection{Benutzerart}
\label{\detokenize{adm/users:benutzerart}}
Damit ein Benutzer sich anmelden kann, muss das Feld Benutzerart
ausgefüllt sein. Die Benutzerart entscheidet vor allem über die
Zugriffsrechte des Benutzers. Welche Benutzerarten es gibt, ist
anwendungsspezifisch.


\subsection{Bearbeitungssperren}
\label{\detokenize{adm/users:bearbeitungssperren}}
In manchen Tabellen darf man nicht einfach wild drauflos bearbeiten,
sondern man muss im Detail-Fenster zuerst auf „Bearbeiten“ klicken, um
die Datenfelder bearbeiten zu können. Das dient dazu, unbeabsichtigte
konkurrierende Änderungen zu vermeiden, d.h. dass zwei Benutzer
gleichzeitig den gleichen Datensatz verändern.

Durch Klick auf „Bearbeiten“ wird eine sogenannte Bearbeitungssperre
angefragt : die Datenfelder, die bisher schreibgeschützt waren, sind
jetzt bearbeitbar. Der Button „Bearbeiten“ hat sich nach „Abbrechen“
verändert. Wenn man auf Speichern klickt, wird diese Sperre
automatisch wieder aufgehoben. Falls man es sich anders überlegt und
seine Änderungen nicht speichern will, sollten man Abbrechen klicken.
Wenn man das unveränderte Formular verlässt, ohne auf Abrrechen zu
klicken, kann es passieren, dass die Sperre aktiv bleibt (weil es für
Lino nicht immer leicht ist rauszufinden, ob du nicht in Wirklichkeit
nur eine Tasse Kaffee trinken gegangen bist).


\chapter{Tipps und Tricks}
\label{\detokenize{tricks:tipps-und-tricks}}\label{\detokenize{tricks::doc}}

\section{\sphinxstyleliteralintitle{\sphinxupquote{Shift+Ctrl+R}}}
\label{\detokenize{tricks:shift-ctrl-r}}
Gewöhne Dir an, wenn Du morgens zum ersten Mal in Lino reingehst, die
Tastenkombination \sphinxcode{\sphinxupquote{Shift+Ctrl+R}} zu drücken. Dadurch sagst du dem
Browser, dass er die Lino-Seite mal komplett neu laden soll.

Auch beim Arbeiten kann es passieren, dass Lino irgendwie nicht wie
gewohnt funktioniert, und in so einer Situation kann ein
\sphinxcode{\sphinxupquote{Shift+Ctrl+R}} Wunder bewirken.

Wenn du in einen Kolonnentitel geklickt hast und dadurch die
Sortierfolge umdefiniert hast, kannst du das am einfachsten wieder
rückgängig machen mit \sphinxcode{\sphinxupquote{Shift+Ctrl+R}}.


\chapter{Änderungen in Lino Tera}
\label{\detokenize{changes/index:anderungen-in-lino-tera}}\label{\detokenize{changes/index:tera-changes}}\label{\detokenize{changes/index::doc}}
See the author’s \sphinxhref{http://luc.lino-framework.org/}{Developer blog}
to get detailed news.  The final truth about what’s going on is only
\sphinxhref{https://github.com/lino-framework/tera}{The Source Code}.


\section{Kommende Version}
\label{\detokenize{changes/coming:kommende-version}}\label{\detokenize{changes/coming:tera-coming}}\label{\detokenize{changes/coming::doc}}
Besprechungsbeginn 2018-10-09.
Release vorgesehen für 2018-11-05.

Offene Entscheidungen:
\begin{itemize}
\item {} 
Wird die Bargeldkasse der Therapeuten abgeschafft?

\item {} 
Kommt eine monatliche Tarifordnung? Zählen die Anwesenheiten auch
für Fakturierung?

\end{itemize}

Aktuelle Fragen:
\begin{itemize}
\item {} 
Werden die Notizen richtig importiert? Momentan kann man Notizen nur
pro Akte anzeigen/erfassen.

\item {} 
Wir hatten ein Fallbeispiel „Kylie B hat ET in Lino, was nicht
stimmt“ gesehen. Das kommt, weil in TIM für Kylie ein PAR mit IdPrt
P steht, weil sie ja als Kind in einer Lebensgruppe mitmacht. Aber
woher soll Lino wissen, dass Kylie keine Einzeltherapie hat sondern
nur als Kind einer anderen Therapie benutzt wird? Okay,
normalerweise haben Personen, die in einer LG oder TG mitmachen,
nicht auch noch eine Einzeltherapie… aber da gibt es bestimmt
Ausnahmen. En attendant haben Fälle wie Kylie in Lino zwei
Therapien.

Ich könnte einen Nachlauf programmieren, der alle ET löscht, deren
Patient auch in anderen Akten Teilnehmer ist.

\item {} 
Beispiel Patient 2070105.  Lino zeigt immer alle Akten an, auch die
stornierten und inaktiven.  Ist das okay?

\end{itemize}

Zeitplan:
\begin{itemize}
\item {} 
\sphinxstylestrong{November 2018} : letzte Arbeiten

\item {} 
\sphinxstylestrong{Dezember 2018} : Daniel, Harry und Gregor beginnen die anderen
Therapeuten zu schulen.  In der Übungsphase sollten alle Endbenutzer
das Erfassen und Verwalten ihrer Akten und Dienstleistungen im Lino
üben.

\item {} 
\sphinxstylestrong{1. Januar 2019} : Umstieg auf Produktionsbetrieb. Ab jetzt werden
keine Daten mehr aus TIM importiert.

\end{itemize}

Entwicklungsschritte:
\begin{itemize}
\item {} 
Die Therapeuten können ihre alltägliche Arbeit
in Lino erledigen wie bisher in TIM.
Akten, Termine und Notizen erfassen und
verwalten.

\item {} 
Das Sekretariat kann Verkaufsrechnungen generieren.

\item {} 
Kalenderplanung.  Lino kann jetzt Terminvorschläge generieren und
hilft bei der Erstellung des Wochenplans.  Es gibt einen Stundenplan
und Ausnahmeregelungen.  Wichtig insbesondere für die Termine im
KITZ.

\end{itemize}

Tagesordnung:
\begin{itemize}
\item {} 
Das, was wir am 2018-10-09 besprochen haben ist fast alles gemacht.
Ich habe nicht jedes Detail hier dokumentiert, weil wir noch in der
iterativen Entwicklungsphase sind.

\item {} 
Dienstleistungen (Termine und Anwesenheiten) werden regelmäßig aus
TIM nach Lino importiert.  In Lino sind sie bisher nur zum Spielen.
Alle Änderungen in Lino gehen beim jeweils nächsten Import verloren.

\item {} 
Statt „Therapie“ sagt Lino jetzt „Akte“, um beim etablierten
Wortschatz bleiben. „Akten“ steht jetzt an erster Stelle im Menü.

\item {} 
Therapien mit zwei Therapeuten haben zur Zeit noch alle Termine
doppelt nach dem Import, weil in TIM jeder Therapeut seine DLS
eingibt. Dubletten kann ich wahrscheinlich automatisch
rausklamüsern, aber bin noch nicht sicher, ob das schwer ist. Zu
klären, ob sich die Arbeit überhaupt lohnt.

\item {} 
Wir haben gesagt, dass bei Akten mit zwei Therapeuten der
Cotherapeut mit den Patienten in der Liste der Teilnehmer kommt.
Ich bin noch nicht sicher, ob euch das gefällt.

\end{itemize}

Falls Zeit bleibt:
\begin{itemize}
\item {} 
Momentan habt ihr nur eine Telefonnummer, GSM-Nr und E-Mail-Adresse
pro Partner. In Lino könnte man auch mehrere „Kontaktdaten“ pro
Partner haben. Daniel und ich haben irgendwann im Juni mal
„beschlossen“, dass eine reicht. Nachteil von mehreren ist, dass die
Bearbeitung dann anders funktioniert als aus TIM gewohnt. Man kann
auch später von single-contact nach multi-contact wechseln, falls
sich rausstellt, dass es Sinn macht.

\item {} 
Could not import zahler 1025, 8, 15, 5, 22, 24

\item {} 
Es gibt in TIM Akten mit ungültigem Tarif 0, 3, 33

\end{itemize}

Aussicht:
\begin{itemize}
\item {} 
Einkaufsrechnungen und Kontoauszüge werden direkt in Lino erfasst.

\item {} 
Verkaufsrechnungen werden noch mit TIM erstellt und ausgedruckt und
dann aus TIM nach Lino importiert.

\item {} 
Buchhaltung 2018 wird parallel auch beim Steuerberater erfasst. Die
Daten in Lino sind da, um zu überprüfen.

\end{itemize}

DONE:

TODO:
\begin{itemize}
\item {} 
Termine mit zwei Therapeuten: einer der beiden ist nur „Assistent“
und wird als Teilnehmer mit Rolle „Cotherapeut“ erfasst.  Für aus
TIM importiere Termine stehen zwei Kalendereinträge in Lino. Diese
könnten bei Bedarf automatisch gelöscht werden.

\item {} 
„Arbeiten als“  zeigt auch den aktuellen User an.

\item {} 
Professional situation : Liste übersetzen. „Homemaker“ ersetzen
durch „Housewife“?

\item {} 
DLS-\textgreater{}IdUser wird scheinbar nicht importiert. Therapeut ist nicht
immer der aus der Akte.

\end{itemize}

TALK
\begin{itemize}
\item {} 
Gekoppelte Termine : für bestimmte Therapien gilt, dass wenn ein
Patient mehrer Termine hintereinander am gleichen Tag hat, diese für
die Rechnung als ein einziger betrachtet werden.  Dieses Konzept
wird überflüssig, falls wir monatliche Abo-Fakturierung einführen.

\item {} 
Abrechnung an Krankenkassen

\item {} 
Klären, wie die Securex-Rechnungen verbucht werden sollen.

\item {} 
Die Kalenderfunktionen soll entweder (1) so gut werden, dass
OpenExchange nicht mehr nötig ist oder (2) mit Kopano synchronisiert
sein.

\item {} 
„endet um“ kann bis auf weiteres leer sein

\item {} 
Abonnements (eine Rechnung für alle Termine einer Therapie)

\item {} 
Tagesplaner

\item {} 
Terminplanung : Wochen-Master (Stundenplan), Monatsplaner (Wo sind
Lücken? Ausnahmen regeln), Wochenansicht mit diversen
Filtermöglichkeiten, Terminblätter drucken zum
Verteilen. Zugewiesene Termine werden nicht angezeigt im Dashboard.

\item {} 
Themen sind pro Familie und pro Klient, Notizen nur pro Klient.

\item {} 
MTI Navigator can be irritating. Possibility to hide certain links \&
conversions. e.g. Person -\textgreater{} Houshold, Person -\textgreater{} Partner should be
hidden for normal users.

\end{itemize}


\section{Baustellenbesichtigung 2018-10-09}
\label{\detokenize{changes/20181009:baustellenbesichtigung-2018-10-09}}\label{\detokenize{changes/20181009:tera-20181009}}\label{\detokenize{changes/20181009::doc}}
Allgemein:
\begin{itemize}
\item {} 
Vera übernimmt die Buchhaltung : Alle Einkaufsrechnungen und
Kontoauszüge werden schon in Lino erfasst.  Verkaufsrechnungen
werden in 2018 noch mit den beiden TIMs erstellt und ausgedruckt,
aber daraufhin die jeweiligen Beträge manuell in Lino erfasst,
evtl. in einem eigenen Journal (ähnlich wie OD), pro Serie von
Rechnungen wird dort ein einziges Dokument erstellt.

\item {} 
Auch die DLS und DLP aus TIM (Termine und Anwesenheiten) werden
jetzt nach Lino importiert.  Ich kann auf Knopfdruck über Nacht neue
Versionen aufspielen und alle Daten aus TIM nach Lino
importieren. Wir haben uns mündlich auf ein Entwicklerpasswort
geeinigt, mit dem DD, LS, GV und HS sich während der Testphase
einloggen können.  Alle anderen Benutzer wurden zwar erstellt, aber
können sich nicht anmelden.

\end{itemize}

DONE:
\begin{itemize}
\item {} 
dashboard aktivieren

\item {} 
Überfällige Termine : nicht schon die von heute, erst ab gestern.

\item {} 
users.UserDetail hat keine Reiter (Dashboard, event\_type, …)

\item {} 
Terminzustand „Unentschuldigt ausgefallen“ fehlt

\item {} 
Wenn DLP-\textgreater{}Status leer ist, dann soll in Lino Anwesend stehen.

\item {} 
Anwesenheiten der Teilnehmer werden nur in therapeutischen Gruppen
erfasst, bei Einzeltherapien und Lebensgruppen gelten immer alle als
anwesend (werden ansonsten gelöscht).  Wenn so ein Termin auf
„Stattgefunden“ gesetzt wird, werden alle Gäste ungefragt auf
„Anwesend“ gesetzt.  Wenn Termin auf Verpasst gesetzt wird, werden
alle auf Unentschuldigt gesetzt.  Lino hat ein Mapping von
EntryStates nach GuestStates. (Details sh. Spezifikationen).

\end{itemize}

TODO:
\begin{itemize}
\item {} 
Preisliste :
- course\_line
- entry\_type
- client\_tariff
- price
- new\_price?

\item {} 
„Arbeiten als“  zeigt auch den aktuellen User an.

\item {} 
Übersetzung „Life groups“

\item {} 
Professional situation : Liste übersetzen. „Homemaker“ ersetzen
durch „Housewife“?

\item {} 
Lebensgruppen haben keine Teilnehmer?

\item {} 
Problem Import Akte Melinda : Mitglieder in Familie sind doppelt. Es
gibt nur eine Familientherapie, keine Einzeltherapie.

\item {} 
DLS-\textgreater{}IdUser wird scheinbar nicht importiert. Therapeut ist nicht
immer der aus der Akte.

\end{itemize}

TALK
\begin{itemize}
\item {} 
Gekoppelte Termine KITZ : wie regeln wir das?

\item {} 
Termine mit zwei Therapeuten: einer der beiden ist nur „Assistent“
und wird als Teilnehmer mit Rolle „Assistent“ erfasst.

\item {} 
Abrechnung an Krankenkassen

\item {} 
Klären, wie die Securex-Rechnungen verbucht werden sollen.

\item {} 
Die Kalenderfunktionen soll so gut werden, dass OpenExchange nicht
mehr nötig ist.

\item {} 
Fakturierung testen: in Lino Rechnungen machen lassen und schauen,
ob sie mit der Wirklichkeit übereinstimmen.

\item {} 
„endet um“ kann bis auf weiteres leer sein

\item {} 
Abonnements (eine Rechnung für alle Termine einer Therapie)

\item {} 
Tagesplaner

\item {} 
Terminplanung : Wochen-Master (Stundenplan), Monatsplaner (Wo sind
Lücken? Ausnahmen regeln), Wochenansicht mit diversen
Filtermöglichkeiten, Terminblätter drucken zum
Verteilen. Zugewiesene Termine werden nicht angezeigt im Dashboard.

\item {} 
Themen sind pro Familie und pro Klient, Notizen nur pro Klient.

\item {} 
MTI Navigator can be irritating. Possibility to hide certain links \&
conversions. e.g. Person -\textgreater{} Houshold, Person -\textgreater{} Partner should be
hidden for normal users.

\end{itemize}


\section{Lino Tera 18.8.16}
\label{\detokenize{changes/20180816:lino-tera-18-8-16}}\label{\detokenize{changes/20180816:tera-18-8-16}}\label{\detokenize{changes/20180816::doc}}
Bei unserer heutigen Baustellenbesichtigung haben wir theoretisch
folgende Punkte abgehakt.

Allgemein:
\begin{itemize}
\item {} 
Nächster Meilenstein ist, dass auch die DLS und DLP aus TIM (Termine
und Anwesenheiten) nach Lino importiert werden und die
Kalenderfunktionen erweitert werden.  En passant kann dann auch die
Fakturierung getestet werden: in Lino Rechnungen machen lassen und
schauen, ob sie mit der Wirklichkeit übereinstimmen.

\item {} 
Wir beginnen bald mit der aktiven Testphase: Ich kann auf Knopfdruck
über Nacht neue Versionen aufspielen und alle Daten aus TIM nach
Lino importieren. Wir haben uns mündlich auf ein Entwicklerpasswort
geeinigt, mit dem DD, LS, GV und HS sich während der Testphase
einloggen können.  Alle anderen Benutzer werden zwar erstellt, aber
können sich nicht anmelden.

\item {} 
Buchhaltung wird von Vera übernommen : Alle Einkaufsrechnungen und
Kontoauszüge werden schon in Lino erfasst.  Verkaufsrechnungen
werden in 2018 noch mit den beiden TIMs erstellt und ausgedruckt,
aber daraufhin die jeweiligen Beträge manuell in Lino erfasst,
evtl. in einem eigenen Journal (ähnlich wie OD), pro Serie von
Rechnungen wird dort ein einziges Dokument erstellt.

\end{itemize}

TODO (Luc):
\begin{itemize}
\item {} 
Professional situation : Liste übersetzen. „Homemaker“ ersetzen
durch „Housewife“?

\item {} 
Doppelte Therapeuten in Auswahlfeld „Benutzer“.

\item {} 
Kein Standardpasswort für die anderen Benutzer.

\item {} 
Terminplanung : Wochen-Master (Stundenplan), Monatsplaner (Wo sind
Lücken? Ausnahmen regeln), Wochenansicht mit diversen
Filtermöglichkeiten, Terminblätter drucken zum
Verteilen. Zugewiesene Termine werden nicht angezeigt im Dashboard.

\item {} 
cal.EntriesByCourse : Übersichtlicher gestalten.  Jahr anzeigen im
Datum.  Im KITZ können mehrere Termine pro Tag stattfinden, an
mehreren Tage pro Woche.  Im SPZ dagegen kommen manche nur 3x pro
Jahr…

\item {} 
Termin erstellen von Therapie aus: geht nicht.

\item {} 
Themen sind pro Familie und pro Klient, Notizen nur pro Klient.

\item {} 
MTI Navigator can be irritating. Possibility to hide certain links \&
conversions. e.g. Person -\textgreater{} Houshold, Person -\textgreater{} Partner should be
hidden for normal users.

\end{itemize}

TALK
\begin{itemize}
\item {} 
tariff : bleibt pro Client und pro Household. ClientTariffs
umbenennen nach TariffGroups oder so. Der eigentliche Stückpreis
(„das Produkt“) steht dann in Enrolment.fee. Dazu brauchen wir
vielleicht noch eine Tabelle von default fees per ClientTariff.

\item {} 
Idee (zu besprechen): Von einem neuen Klienten aus könnte man eine
Aktion starten, die eine Notiz fürs Erstgespräch erstellt, wobei
Lino dann falls nötig automatisch eine Aktivität erstellt.

\item {} 
Übersetzung ClientStates : Statt „Zustand“ eines Patienten „Stand
der Beratung“. Aber wir haben in Lino ein Feld „Zustand“ an vielen
Stellen: pro Therapie, pro Patient, pro Einschreibung, pro
Anwesenheit.  Ich zögere noch, die alle nach „Stand“ umzubenennen.

\item {} 
Übersetzung cal.EntryType „Kalendereintragsart“ ersetzen durch
„Dienstleistungsart“.  Problem: das stimmt nicht ganz, denn es
werden auch z.B. „Feiertage“ oder „Teamversammlungen“ kommen, die
dann eindeutig \sphinxstyleemphasis{keine} Dienstleistungen sind.

\item {} 
Dass eine Therapie auch für einen bestimmten Haushalt (nicht
Klienten) sein kann, macht die Sache etwas kompliziert.  so was wäre
mit einer Standardsoftware sicherl nicht machbar.

\item {} 
Was in TIM unter „Stand der Beratung“ stand, steht in Lino jetzt
unter EnrolmentStates:

\fvset{hllines={, ,}}%
\begin{sphinxVerbatim}[commandchars=\\\{\}]
\PYG{l+m+mi}{01} \PYG{n}{dauert} \PYG{n}{an}
\PYG{l+m+mi}{03} \PYG{n}{abgeschlossen}
\PYG{l+m+mi}{05} \PYG{n}{automatisch} \PYG{n}{abgeschlossen}
\PYG{l+m+mi}{06} \PYG{n}{Abbruch} \PYG{n}{der} \PYG{n}{Beratung}
\PYG{l+m+mi}{09} \PYG{n}{Weitervermittlung}
\PYG{l+m+mi}{12} \PYG{n}{nur} \PYG{n}{Erstkontakt}
\end{sphinxVerbatim}

\item {} 
in TIM hatten wir das Feld „Stand der Akte“ pro „Partner“. In Lino
werden aus den „Partnern“ aber zwei verschiedene Dinge: „Therapien“
(genauer gesagt „Einschreibungen in einer Therapie“) und „Patienten“
(einmaliger Datensatz pro physischer Person). Folglich gibt es
jetzt zwei verschiedene Felder: „Stand der Einschreibung“ und „Stand
der Patientenakte“. In beiden Feldern habe ich momentan die
Auswahlliste „Stand der Akte“ aus TIM.

\item {} 
ClientStates: das Feld bleibt pro Patient und pro Haushalt, aber
kriegt nach Abschluss der tl2.py-Phase vielleicht neue Werte (Aktiv,
Inaktiv)

\end{itemize}

TODO (Vera)
\begin{itemize}
\item {} 
Partner, die seit Ende 2017 in TIM erstellt wurden, müssen auch in
Lino erstellt werden.

\end{itemize}

DONE (to verify):
\begin{itemize}
\item {} 
Notizen und Themen werden jetzt importiert aus TIM.

\item {} 
Notizen und Themen sind sehr vertraulich (nur für Therapeuten),
Termine werden auch vom Sekretariat gesehen.

\item {} 
Site.languages : auch EN und NL

\item {} 
Kalendereintragsart war leer. Jetzt sind alle DLA importiert und
jede DLS hat eine DLA.

\item {} 
DLA aus TIM importieren nach cal.EventType (Kalendereintragsart).

\item {} 
No de GSM, Date naissance, Geschlecht n’ont pas été importés

\item {} 
birth\_date wird jetzt importiert

\item {} 
Therapie E130280 : nicht Harry sondern Daniel müsste Therapeut
sein. Falsch importiert. Lino nahm prioritär den T1 statt des T2.

\item {} 
Rechnungsempfänger und Krankenkasse importieren : pro Patient, nicht
pro Einschreibung.

\item {} 
Akten E180246 und E180247 fehlen in Lino.

\item {} 
Notizen sind nur bis November 2017 importiert worden

\item {} 
Status der importierten Anwesenheiten war immer leer.  Status
„Verpasst“ heißt „abwesend“ in Lino.

\end{itemize}

DONE and verified:
\begin{itemize}
\item {} 
Teilnehmer der Gruppentherapien fehlten noch.

\item {} 
Das Alter wird bei Kleinkindern auf z.B. „18 Monate“

\item {} 
Therapeutische Gruppen : Kolonnenlayout

\item {} 
Kindergruppe 2016 hat keine Therapie in Lino. Kindergruppe 2018
fehlt komplett.  „Psychodrama Do 2018“ hat Anwesenheiten pro
Teilnehmer in TIM korrekt.

\item {} 
Fakturationsadresse sichtbar machen in Ansicht Patient und Haushalt.

\item {} 
Import Fakturationsadresse (Zahler) aus TIM scheint nicht zu funktionieren.

\item {} 
Klientenkontaktarten : Liste füllen und auch Daten importieren aus
TIM (z.B. Krankenkasse)

\item {} 
nationality ist leer. Es fehlen viele Länder.

\item {} 
Übersetzung Enrolment = „Teilnahme“ (nicht „Einschreibung“)

\item {} 
NotesByPatient raus. Auch „Health situation“ und zwei weitere
Memofelder.

\item {} 
Modul humanlinks raus, phones rein.

\item {} 
„Alte Daten löschen“ in TIM

\end{itemize}

Außerdem besprochen:
\begin{itemize}
\item {} 
Es wird eine Serie „virtueller Personen“ geben, z.B. „Kaleido“ : das
bedeutet „irgendein Mitarbeiter von Kaleido, der als Vertreter
fungiert“. Wer das jeweils genau ist, wird nicht in Lino notiert.

\item {} 
Raum einer Therapie (eines Termins)? Bleibt.

\item {} 
Brauchen wir eine weitere Tabelle von „Anfragen“ bzw. „Projekten“?
Vorerst nicht.

\item {} 
Notiz Erstgespräch (Create a note from patient without therapy) :
Meine Idee („Von einem neuen Klienten aus könnte man eine Aktion
starten, die eine Notiz fürs Erstgespräch erstellt, wobei Lino dann
falls nötig automatisch eine Aktivität erstellt“) ist nicht
nötig.  Stattdessen kommt NotesByPatient komplett raus. Notizen sieht
man nur über die jeweilige Therapie.

\item {} 
Pro Therapie gibt es einen verantwortlichen Therapeuten. Die
„Disziplinen“ im KITZ werden als unabhängige Therapien erfasst.  Das
Erstgespräch bzw. die Testphase gilt ebenfalls als eine eigene
Therapie.  Der Therapeut dieser Therapie ist zunächst auch
Primärbegleiter.

\end{itemize}


\chapter{tim2lino}
\label{\detokenize{tim2lino:tim2lino}}\label{\detokenize{tim2lino::doc}}
Der \sphinxstylestrong{Tarif} einer Akte aus TIM kommt in Lino nach \sphinxtitleref{Enrolment.fee}.

Der \sphinxstylestrong{Zahler} einer Akte kommt in Lino nach Partner.invoice\_decipient
und nach Course.partner.


\chapter{Legale Hinweise}
\label{\detokenize{legal:legale-hinweise}}\label{\detokenize{legal::doc}}
\sphinxstylestrong{Verantwortlicher Herausgeber} dieser Webseite ist Luc Saffre, Uus
tn. 1, Vana-Vigala küla, 78003 Raplamaa, Estland.

\sphinxstylestrong{Datenschutzerklärung} : Diese Website verwendet keine Cookies und
sammelt keine Daten über ihre Besucher.



\renewcommand{\indexname}{Stichwortverzeichnis}
\printindex
\end{document}